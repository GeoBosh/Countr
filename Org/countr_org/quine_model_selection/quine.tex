% Created 2017-11-15 Wed 07:16
% Intended LaTeX compiler: pdflatex
\documentclass[a4paper,twoside,11pt]{article}
                              \usepackage[T1]{fontenc}
\usepackage{libertine}
\renewcommand*\oldstylenums[1]{{\fontfamily{fxlj}\selectfont #1}}
\usepackage{lmodern}
\usepackage[margin=1.0in]{geometry}
\usepackage{rotating}
\usepackage{setspace,amsmath,amsfonts,amssymb,bm}
\usepackage{graphicx}
\usepackage[usenames,dvipsnames]{xcolor}
\definecolor{Red}{rgb}{0.5,0,0}
\definecolor{NavyBlue}{rgb}{0.1,0.1,0.45}
\definecolor{MidnightBlue}{rgb}{0.1,0.1,0.65}
\usepackage[pdftex,hyperfootnotes=true,plainpages=false,pdfpagelabels,hypertexnames=true,naturalnames,pdfproducer={Latex},pdfcreator={pdflatex},bookmarks,bookmarksnumbered,colorlinks,linkcolor=MidnightBlue,citecolor=NavyBlue,filecolor=black,urlcolor=Red,breaklinks=true]{hyperref}
\usepackage{authblk}
\usepackage{xspace,helvet}
\usepackage{moreverb}
\usepackage{url, booktabs}
\usepackage{cleveref}
\usepackage[pdftex]{lscape}
\usepackage{fullpage}
\usepackage{booktabs}
\usepackage{multirow}
\usepackage{rotating}
\usepackage{pifont}
\usepackage{setspace}
\usepackage{threeparttable}
\usepackage{tabulary}
\usepackage[toc,page]{appendix}
\usepackage{pbox}
\usepackage[font=small]{caption}
\newcommand{\rom}[1]{\uppercase\expandafter{\romannumeral #1\relax}}
\newcommand{\E}{\mathsf{E}}
\newcommand{\VAR}{\mathsf{VAR}}
\newcommand{\COV}{\mathsf{COV}}
\newcommand{\Prob}{\mathsf{P}}
\newcommand{\RNum}[1]{\uppercase\expandafter{\romannumeral #1\relax}}
\newcommand{\dee}{\,\mbox{d}}
\newcommand{\naive}{na\"{\i}ve }
\newcommand{\eg}{e.g.\xspace}
\newcommand{\ie}{i.e.\xspace}
\newcommand{\pdf}{pdf.\xspace}
\newcommand{\etc}{etc.\@\xspace}
\newcommand{\PhD}{Ph.D.\xspace}
\newcommand{\MSc}{M.Sc.\xspace}
\newcommand{\BA}{B.A.\xspace}
\newcommand{\R}{\texttt{R}\xspace}
\usepackage{paralist}
\let\itemize\compactitem
\let\description\compactdesc
\let\enumerate\compactenum
\let\enumerate\inparaenum
\renewenvironment{enumerate}{\begin{inparaenum}[(i)]}{\end{inparaenum}}
\renewenvironment{enumerate}{\begin{inparaenum}[(a)]}{\end{inparaenum}}
\usepackage[round]{natbib}
\author[1]{Tarak Kharrat}
\author[2]{Georgi N. Boshnakov}
\affil[1]{Salford Business School, University of Salford, UK.}
\affil[2]{School of Mathematics, University of Manchester, UK.}
\newcommand{\code}{\texttt}
\newcommand{\CountrDist}[1]{\texttt{"#1"}}
\date{\today}
\title{Model selection and comparison\\\medskip
\large an example with package \texttt{Countr}}
\hypersetup{
 pdfauthor={},
 pdftitle={Model selection and comparison},
 pdfkeywords={},
 pdfsubject={},
 pdfcreator={Emacs 25.2.1 (Org mode 9.1.2)}, 
 pdflang={English}}
\begin{document}

\maketitle
\begin{abstract}
This document describes a strategy to choose between various possible
count models. The computation described in this document is done in \texttt{R}
\citep{Rcore} using the contributed package \texttt{Countr} \citep{RpackageCountr} and
the \texttt{quine} data shipped with the \texttt{MASS} package \citep{Venables20102MASS}. The
ideas used here are inspired by the demand for medical care example detailed in
\citet[Section 6.3]{cameron2013regression}. 

This vignette is part of package \texttt{Countr} \citep[see][]{CountrJssArticle}.
\end{abstract}


\section{Prerequisites}
\label{sec:orga5a3f6d}

We will do the analysis of the data with package \texttt{Countr}, so we load it:
\begin{verbatim}
library(Countr)
library("MASS") # for glm.nb()
\end{verbatim}

Packages \texttt{dplyr} \citep{dplyr2016} and \texttt{xtable} \citep{xtable2016} provide
usefull facilities for data manipulation and presentation:
\begin{verbatim}
library(dplyr) 
library(xtable)
\end{verbatim}

\section{Data}
\label{sec:org9e0589a}

The dataset used in this example is the \texttt{quine} data shipped with 
package \texttt{MASS} \citep{Venables20102MASS} and first analysed in
\citet{aitkin1978analysis}. The data can be loaded in the usual way:
\begin{verbatim}
data(quine, package = "MASS")
\end{verbatim}

The dataset gives the number of days absent from school (\texttt{Days}) of
146
children in a particular school year. A number of
explanatory variables are available describing the children's ethnic background
(\texttt{Eth}), sex (\texttt{Sex}), age (\texttt{Age}) and learner status
(\texttt{Lrn}).  The count variable \texttt{Days} is characterised by large
\emph{overdispersion} --- the variance is more than
16 times larger the mean, 
264.2
versus 
16.46.
Table \ref{tab:quine:days} gives an idea about its
distribution. The entries in the table were calculated as follows:


\begin{verbatim}
breaks_ <- c(0, 1, 3, 5:7, 9, 12, 15, 17, 23, 27, 32)
freqtable <- 
    count_table(count = quine$Days, breaks = breaks_, formatChar = TRUE)
\end{verbatim}



\begin{table}
\centering
\begin{tabular}{lrrrrrrrrr}

% latex table generated in R 3.4.2 by xtable 1.8-2 package
% Tue Nov 14 16:37:27 2017
 & 0 & 1-2 & 3-4 & 5 & 6 & 7-8 & 9-11 \\ 
  \hline
Frequency & 9 & 10 & 7 & 19 & 8 & 10 & 13 \\ 
  Relative frequency & 0.062 & 0.068 & 0.048 & 0.13 & 0.055 & 0.068 & 0.089 \\ 
   \hline

\\[5pt]
% latex table generated in R 3.4.2 by xtable 1.8-2 package
% Tue Nov 14 16:37:27 2017
 & 12-14 & 15-16 & 17-22 & 23-26 & 27-31 & $>$= 32 \\ 
  \hline
Frequency & 13 & 6 & 14 & 6 & 6 & 25 \\ 
  Relative frequency & 0.089 & 0.041 & 0.096 & 0.041 & 0.041 & 0.17 \\ 
   \hline

\end{tabular}
\caption{quine data: Frequency distribution of column \texttt{Days}.}
\label{tab:quine:days}
\end{table}



\section{Models for \texttt{quine} data}
\label{sec:orgd1e9d6d}

Given the extreme over-dispersion observed in the \texttt{quine} data, we do not expect
the Poisson data to perform well. Nevertheless, we can still use this model as a
starting point and treat it as a benchmark (any model worse than Poisson should
be strongly rejected). Besides, the negative binomial (as implemented in
\texttt{MASS:glm.nb()}) and 3 renewal-count models based on weibull, gamma and
generalised-gamma inter-arrival times are also considered giving in total 5
models to choose from. The code used to fit the 5 models is provided below:
\begin{verbatim}
quine_form <- as.formula(Days ~ Eth + Sex + Age + Lrn)
pois <- glm(quine_form, family = poisson(), data = quine)
nb <- glm.nb(quine_form, data = quine)

## various renewal models
wei <- renewalCount(formula = quine_form, data = quine, dist = "weibull",
                          computeHessian = FALSE, weiMethod = "conv_dePril",
                          control = renewal.control(trace = 0,
                              method = c("nlminb", 
                                  "Nelder-Mead","BFGS")),
                          convPars = list(convMethod = "dePril")
                          )

gam <- renewalCount(formula = quine_form, data = quine, dist = "gamma",
                    computeHessian = FALSE, weiMethod = "conv_dePril",
                    control = renewal.control(trace = 0,
                                              method = "nlminb"),
                        convPars = list(convMethod = "dePril")
                    )

gengam <- renewalCount(formula = quine_form, data = quine, dist = "gengamma",
                       computeHessian = FALSE, weiMethod = "conv_dePril",
                       control = renewal.control(trace = 0,
                                                 method = "nlminb"),
                           convPars = list(convMethod = "dePril")
                       )
\end{verbatim}

Note that it is possible to try several optimisation algorithms in a single call to
\texttt{renewalCount()}, as in the computation of \texttt{wei} above  for the weibull-count model. 
The function will choose the best performing one in terms of
value of the objective function (largest log-likelihood). It also reports the
computation time  of each method, which may be useful for future computations
(for example, bootstrap).


\section{Model Selection and Comparison}
\label{sec:org084a744}

The different models considered here are fully parametric. Therefore, a
straightforward method to use to discriminate between models is a likelihood
ratio test (LR). This is possible when models are nested and in this case the LR
statistic has the usual \(\chi^2(p)\) distribution, where \(p\) is the difference in
the number of parameters in the model. In this current study, we can compare all
the renewal-count models against Poisson, negative-binomial against Poisson,
weibull-count against generalised-gamma and gamma against the generalised-gamma.

For non-nested models, the standard approach is to use information criteria such
as the Akaike information criterion (AIC) and the Bayesian information criterion
(BIC). This method can be applied to discriminate between weibull and gamma
renewal count models and between these two models and the negative binomial.

Therefore, a possible strategy can be summarised as follows:
\begin{itemize}
\item Use the LR test to compare Poisson with negative binomial.
\item Use the LR test to compare Poisson with weibull-count.
\item Use the LR test to compare Poisson with gamma-count.
\item Use the LR test to compare Poisson with generalised-gamma-count.
\item Use the LR test to compare weibull-count with generalised-gamma-count.
\item Use the LR test to compare gamma-count with generalised-gamma-count.
\item Use information criteria to compare gamma-count with weibull-count.
\item Use information criteria to compare weibull-count to negative binomial.
\end{itemize}

% latex table generated in R 3.4.1 by xtable 1.8-2 package
% Wed Nov 15 07:06:47 2017
\begin{table}[ht]
\centering
\begin{tabular}{rlrrr}
  \hline
 & Alternative.model & Chisq & Df & Critical\_value \\ 
  \hline
1 & negative-binomial & 1192.03 & 1.00 & 3.84 \\ 
  2 & weibull & 1193.21 & 1.00 & 3.84 \\ 
  3 & gamma & 1193.36 & 1.00 & 3.84 \\ 
  4 & generalised-gamma & 1194.46 & 2.00 & 5.99 \\ 
   \hline
\end{tabular}
\caption{LR results against Poisson model. Each row compares an alternative model vs the Poisson model. All alternatives are preferable to Poisson.} 
\label{tab:lr_pois}
\end{table}

As observed in Table \ref{tab:lr_pois}, the LR test rejects the null
hypothesis and all the alternative models are preferred to Poisson. This due to
the large over-dispersion discussed in the previous section. Next, we use the LR
test to discriminate between the renewal count models:
% latex table generated in R 3.4.1 by xtable 1.8-2 package
% Wed Nov 15 07:06:48 2017
\begin{table}[ht]
\centering
\begin{tabular}{rlrrr}
  \hline
 & Model & Chisq & Df & Critical\_value \\ 
  \hline
1 & weibull & 1.25 & 1.00 & 3.84 \\ 
  2 & gamma & 1.10 & 1.00 & 3.84 \\ 
   \hline
\end{tabular}
\caption{LR results against generalised-gamma model} 
\label{tab:lr_gengam}
\end{table}

The results of Table \ref{tab:lr_gengam} suggest that the weibull and gamma models
should be preferred to the generalised gamma model.

Finally, we use information criteria to choose the best model among the weibull
and gamma renewal models and the negative binomial:
\begin{verbatim}
ic <- data.frame(Model = c("weibull", "gamma", "negative-binomial"),
                 AIC = c(AIC(wei), AIC(gam), AIC(nb)),
                 BIC = c(BIC(wei), BIC(gam), BIC(nb)),
                 stringsAsFactors = FALSE                    
                 )                  
                         
print(xtable(ic, caption = "Information criteria results", 
             label = "tab:ic_models"))                        
\end{verbatim}

% latex table generated in R 3.4.2 by xtable 1.8-2 package
% Tue Nov 14 16:51:36 2017
\begin{table}[ht]
\centering
\begin{tabular}{rlrr}
  \hline
 & Model & AIC & BIC \\ 
  \hline
1 & weibull & 1107.98 & 1131.84 \\ 
  2 & gamma & 1107.83 & 1131.70 \\ 
  3 & negative-binomial & 1109.15 & 1133.02 \\ 
   \hline
\end{tabular}
\caption{Information criteria results} 
\label{tab:ic_models}
\end{table}

It is worth noting here that all models have the same degree of freedom and
hence comparing information criteria is equivalent to comparing the likelihood
value. According to Table \ref{tab:ic_models}, the gamma renewal model is slightly
preferred to the weibull model.   

\section{Goodness-of-fit}
\label{sec:org2626954}
We conclude this analysis by running a formal chi-square goodness of fit test 
\citep[Section 5.3.4]{cameron2013regression} to the results of the previously
selected model.
\begin{verbatim}
  gof <- chiSq_gof(gam, breaks = breaks_)
  print(gof)
\end{verbatim}

\begin{verbatim}
chi-square goodness-of-fit test

Cells considered 0 1-2 3-4 5 6 7-8 9-11 12-14 15-16 17-22 23-26 27-31 >= 32
  DF  Chisq Pr(>Chisq)
1 12 17.502     0.1317
\end{verbatim}

The null hypothesis cannot be rejected at standard confidence levels and
conclude that the selected model presents a good fit to the data. User can check
that the same test yields similar results for the weibull and negative binomial
models but comfortably rejects the null hypothesis for the Poisson and
generalised gamma models. These results confirm what we observed in the previous
section.

\label{sec:orga3dd32d}
We conclude this document by saving the work space to avoid re-running the
computation in future exportation of the document:
\begin{verbatim}
save.image()
\end{verbatim}




\label{sec:orgdf2f41b}
\bibliographystyle{apalike}
\bibliography{quine}
\end{document}
