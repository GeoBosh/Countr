% Created 2017-10-20 Fri 19:36
% Intended LaTeX compiler: pdflatex
\documentclass[a4paper,twoside,11pt]{article}
                              \usepackage[T1]{fontenc}
\usepackage{libertine}
\renewcommand*\oldstylenums[1]{{\fontfamily{fxlj}\selectfont #1}}
\usepackage{lmodern}
\usepackage[margin=1.0in]{geometry}
\usepackage{rotating}
\usepackage{setspace,amsmath,amsfonts,amssymb,bm}
\usepackage{graphicx}
\usepackage[usenames,dvipsnames]{xcolor}
\definecolor{Red}{rgb}{0.5,0,0}
\definecolor{NavyBlue}{rgb}{0.1,0.1,0.45}
\definecolor{MidnightBlue}{rgb}{0.1,0.1,0.65}
\usepackage[pdftex,hyperfootnotes=true,plainpages=false,pdfpagelabels,hypertexnames=true,naturalnames,pdfproducer={Latex},pdfcreator={pdflatex},bookmarks,bookmarksnumbered,colorlinks,linkcolor=MidnightBlue,citecolor=NavyBlue,filecolor=black,urlcolor=Red,breaklinks=true]{hyperref}
\usepackage{authblk}
\usepackage{xspace,helvet}
\usepackage{moreverb}
\usepackage{url, booktabs}
\usepackage{cleveref}
\usepackage[pdftex]{lscape}
\usepackage{fullpage}
\usepackage{booktabs}
\usepackage{multirow}
\usepackage{rotating}
\usepackage{pifont}
\usepackage{setspace}
\usepackage{threeparttable}
\usepackage{tabulary}
\usepackage[toc,page]{appendix}
\usepackage{pbox}
\usepackage[font=small]{caption}
\newcommand{\rom}[1]{\uppercase\expandafter{\romannumeral #1\relax}}
\newcommand{\E}{\mathsf{E}}
\newcommand{\VAR}{\mathsf{VAR}}
\newcommand{\COV}{\mathsf{COV}}
\newcommand{\Prob}{\mathsf{P}}
\newcommand{\RNum}[1]{\uppercase\expandafter{\romannumeral #1\relax}}
\newcommand{\dee}{\,\mbox{d}}
\newcommand{\naive}{na\"{\i}ve }
\newcommand{\eg}{e.g.\xspace}
\newcommand{\ie}{i.e.\xspace}
\newcommand{\pdf}{pdf.\xspace}
\newcommand{\etc}{etc.\@\xspace}
\newcommand{\PhD}{Ph.D.\xspace}
\newcommand{\MSc}{M.Sc.\xspace}
\newcommand{\BA}{B.A.\xspace}
\newcommand{\R}{\texttt{R}\xspace}
\usepackage{paralist}
\let\itemize\compactitem
\let\description\compactdesc
\let\enumerate\compactenum
\let\enumerate\inparaenum
\renewenvironment{enumerate}{\begin{inparaenum}[(i)]}{\end{inparaenum}}
\renewenvironment{enumerate}{\begin{inparaenum}[(a)]}{\end{inparaenum}}
\usepackage[round]{natbib}
\author[1]{Tarak Kharrat}
\author[2]{Georgi N. Boshnakov}
\affil[1]{Salford Business School, University of Salford, UK.}
\affil[2]{School of Mathematics, University of Manchester, UK.}
\newcommand{\code}{\texttt}
\date{}
\title{Notes about Org\\\medskip
\large with examples}
\hypersetup{
 pdfauthor={},
 pdftitle={Notes about Org},
 pdfkeywords={},
 pdfsubject={},
 pdfcreator={Emacs 25.2.1 (Org mode 9.1.2)}, 
 pdflang={English}}
\begin{document}

\maketitle
These notes are when working with Org version 9.1.2.

\section{General notes}
\label{sec:org27e96d4}

\begin{itemize}
\item If the code of the chunks is evaluated when exporting the file, e.g. \texttt{C-c C-e
  l p} , the results in the org file are not updated (the updated
stuff appears in the exported document). Maybe this depends on options but we
better be aware of this.

Note that the results of calculation of inline code are inserted in the
buffer.

\item 


\item TODO: use the facility from main.pdf to provide visual distinction between
code and results.

\item Exporting nothing (\texttt{:exports none})
\end{itemize}

Here is a chunk that loads \texttt{xtable} but has \texttt{:exports none} and should not
appear. 
This is currently unpredictable. Even during a single session it may reappear!

\begin{itemize}
\item I changed the inline blocks to return raw values (for the moment), since it
seems better that the numbers are passed to \TeX{} wthout being wrapped in
texttt. 

However 'raw' has the unfortunate effect that it doesn't wrap the result of
the inline block in \ldots{}, so repeated evaluation will produce
repeated values. 

\textbf{TODO} - there may be some option to avoid this, otherwise 'raw' may have to
be removed. (but see below for a workflow that woeks until the whole file is
recalculated.

\item I adopted the following workflow while editing:

i. Most of the time, I simply export the (\texttt{C-c C-e l p}). This recomputed the
inline blocks but doesn't put the results in the org buffer, only in the
exported \TeX{} file. 

ii. When I change a chunk, I recalculate only that chunk (\texttt{C-c C-c}).
 Then export as in i.

\textbf{TODO}: This will fail for objects with the same names are at different
places in the doc. Change them.

The trouble is that if the whole file is recomputed (\texttt{C-c C-v C-b}) 
the inline chunks with the 'raw' option are computed and the results put in
the file (as noted above). Then during export all such results are
duplicated.
\end{itemize}


\section{Formatting tables}
\label{sec:org0789c0f}

Org provides some formatting and produces tables from matrices and data frames
but seems to ignore the column names and may be all attributes. 

We will use \texttt{xtable} for some printing.
\begin{verbatim}
library(xtable)
\end{verbatim}



Here we print (a subset of) a data frame, as \texttt{R} would print it on the console.
We add a character column to be sure that this works with mixed type columns.
\begin{verbatim}
fewCars <- head(cars)
fewCars$char <- letters[1:nrow(fewCars)]
fewCars$num <- 1:nrow(fewCars) * pi
fewCars
\end{verbatim}

\begin{verbatim}
  speed dist char       num
1     4    2    a  3.141593
2     4   10    b  6.283185
3     7    4    c  9.424778
4     7   22    d 12.566371
5     8   16    e 15.707963
6     9   10    f 18.849556
\end{verbatim}

A proper table can be obtained using \texttt{xtable}. 
\begin{verbatim}
xfew <- xtable(fewCars, digits = 5)
print(xfew, floating = FALSE)
\end{verbatim}

% latex table generated in R 3.4.1 by xtable 1.8-2 package
% Fri Oct 20 17:47:42 2017
\begin{tabular}{rrrlr}
  \hline
 & speed & dist & char & num \\ 
  \hline
1 & 4.00000 & 2.00000 & a & 3.14159 \\ 
  2 & 4.00000 & 10.00000 & b & 6.28319 \\ 
  3 & 7.00000 & 4.00000 & c & 9.42478 \\ 
  4 & 7.00000 & 22.00000 & d & 12.56637 \\ 
  5 & 8.00000 & 16.00000 & e & 15.70796 \\ 
  6 & 9.00000 & 10.00000 & f & 18.84956 \\ 
   \hline
\end{tabular}

It is worth storing the result of calling \texttt{xtable} in a variable and printing it
separately, since the function \texttt{xtable()} has one set of options, while the
print method for objects of type \texttt{xtable} has other options.  For example:
\begin{verbatim}
xfew <- xtable(fewCars, caption = "Information about some cars.")
print(xfew)
\end{verbatim}

% latex table generated in R 3.4.1 by xtable 1.8-2 package
% Fri Oct 20 17:47:42 2017
\begin{table}[ht]
\centering
\begin{tabular}{rrrlr}
  \hline
 & speed & dist & char & num \\ 
  \hline
1 & 4.00 & 2.00 & a & 3.14 \\ 
  2 & 4.00 & 10.00 & b & 6.28 \\ 
  3 & 7.00 & 4.00 & c & 9.42 \\ 
  4 & 7.00 & 22.00 & d & 12.57 \\ 
  5 & 8.00 & 16.00 & e & 15.71 \\ 
  6 & 9.00 & 10.00 & f & 18.85 \\ 
   \hline
\end{tabular}
\caption{Information about some cars.} 
\end{table}
\end{document}
