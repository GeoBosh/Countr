% Created 2017-10-24 Tue 12:06
% Intended LaTeX compiler: pdflatex
\documentclass[a4paper,twoside,11pt]{article}
                              \usepackage[T1]{fontenc}
\usepackage{libertine}
\renewcommand*\oldstylenums[1]{{\fontfamily{fxlj}\selectfont #1}}
\usepackage{lmodern}
\usepackage[margin=1.0in]{geometry}
\usepackage{rotating}
\usepackage{setspace,amsmath,amsfonts,amssymb,bm}
\usepackage{graphicx}
\usepackage[usenames,dvipsnames]{xcolor}
\definecolor{Red}{rgb}{0.5,0,0}
\definecolor{NavyBlue}{rgb}{0.1,0.1,0.45}
\definecolor{MidnightBlue}{rgb}{0.1,0.1,0.65}
\usepackage[pdftex,hyperfootnotes=true,plainpages=false,pdfpagelabels,hypertexnames=true,naturalnames,pdfproducer={Latex},pdfcreator={pdflatex},bookmarks,bookmarksnumbered,colorlinks,linkcolor=MidnightBlue,citecolor=NavyBlue,filecolor=black,urlcolor=Red,breaklinks=true]{hyperref}
\usepackage{authblk}
\usepackage{xspace,helvet}
\usepackage{moreverb}
\usepackage{url, booktabs}
\usepackage{cleveref}
\usepackage[pdftex]{lscape}
\usepackage{fullpage}
\usepackage{booktabs}
\usepackage{multirow}
\usepackage{rotating}
\usepackage{pifont}
\usepackage{setspace}
\usepackage{threeparttable}
\usepackage{tabulary}
\usepackage[toc,page]{appendix}
\usepackage{pbox}
\usepackage[font=small]{caption}
\newcommand{\rom}[1]{\uppercase\expandafter{\romannumeral #1\relax}}
\newcommand{\E}{\mathsf{E}}
\newcommand{\VAR}{\mathsf{VAR}}
\newcommand{\COV}{\mathsf{COV}}
\newcommand{\Prob}{\mathsf{P}}
\newcommand{\RNum}[1]{\uppercase\expandafter{\romannumeral #1\relax}}
\newcommand{\dee}{\,\mbox{d}}
\newcommand{\naive}{na\"{\i}ve }
\newcommand{\eg}{e.g.\xspace}
\newcommand{\ie}{i.e.\xspace}
\newcommand{\pdf}{pdf.\xspace}
\newcommand{\etc}{etc.\@\xspace}
\newcommand{\PhD}{Ph.D.\xspace}
\newcommand{\MSc}{M.Sc.\xspace}
\newcommand{\BA}{B.A.\xspace}
\newcommand{\R}{\texttt{R}\xspace}
\usepackage{paralist}
\let\itemize\compactitem
\let\description\compactdesc
\let\enumerate\compactenum
\let\enumerate\inparaenum
\renewenvironment{enumerate}{\begin{inparaenum}[(i)]}{\end{inparaenum}}
\renewenvironment{enumerate}{\begin{inparaenum}[(a)]}{\end{inparaenum}}
\usepackage[round]{natbib}
\author[1]{Tarak Kharrat}
\author[2]{Georgi N. Boshnakov}
\affil[1]{Salford Business School, University of Salford, UK.}
\affil[2]{School of Mathematics, University of Manchester, UK.}
\newcommand{\code}{\texttt}
\newcommand{\CountrDist}[1]{\texttt{"#1"}}
\date{}
\title{Comparing the performance of computation methods\\\medskip
\large an example with package \texttt{Countr} and the \texttt{fertility} data}
\hypersetup{
 pdfauthor={tarak},
 pdftitle={Comparing the performance of computation methods},
 pdfkeywords={},
 pdfsubject={},
 pdfcreator={Emacs 25.2.2 (Org mode 9.0.7)}, 
 pdflang={English}}
\begin{document}

\maketitle
\begin{abstract}
This short document compares the performance of the different algorithms
implemented in \texttt{Countr} to fit renewal-count models. The computation described
here is based on the \texttt{fertility} data shipped with the package and the
weibull-count model which allows using series based methods on top of the other
convolution methods. More details about the different computation methods can be
found in \citet{baker2017event}.

This vignette is part of package \texttt{Countr} \citep[see][]{CountrJssArticle}.
\end{abstract}


\section{Prerequisites}
\label{sec:org6deafec}

We will do the analysis of the data with package \texttt{Countr}, so we load it:
\begin{verbatim}
library(Countr)
\end{verbatim}

Package \texttt{rebenchmark} \citep{rbench2012} will also be used here to facilitate
performance computation
\begin{verbatim}
library(rbenchmark) 
\end{verbatim}

\section{Comparing performance of different methods}
\label{sec:org7674dd3}
The data used here is the \texttt{fertility} data shipped with the package and
described in length in \citet{winkelmann1995duration}. 

The execution time depends obviously on the machine used but the relative order
should remain the same regardless of the characteristic of your machine. The
\texttt{benchmark()} routine from the \texttt{rbenchmark} package \citet{rbench2012} will be
used to compare the different computation methods discussed in
\citet{baker2017event}. We selected the parameters for every method such as we
achieve an error or at least \(10^{-8}\). The code below reproduces the results
reported in \citet[Table 2]{baker2017event}. As a benchmark, we use an
adaptation of the original code gently provided by Blake McShane 
\citep{mcshane2008count}. The McShane's code is implemented in a separate file
and is not shown here. We repeat the iterations 1000 times.
\begin{verbatim}
source("mcShaneCode.R")
data(fertility)

## config: choose parameters (selection process not shown here)
children <- fertility$children
shape <- 1.116
scale <- rep(2.635, length(children))
rep <- 1000
nstepsConv <- c(132, 24, 132, 24, 132, 36)
ntermsSeries <- c(20, 17)
conv_series_acc <- 1e-7

## performance model
perf <- benchmark(direct0 =
                      dWeibullCount_loglik(children, shape, scale, "conv_direct",
                                           1, TRUE, nstepsConv[1],
                                           conv_extrap = FALSE),
                  direct1 =
                      dWeibullCount_loglik(children, shape, scale, "conv_direct",
                                           1, TRUE, nstepsConv[2],
                                           conv_extrap = TRUE),
                  naive0 =
                      dWeibullCount_loglik(children, shape, scale, "conv_naive",
                                           1, TRUE, nstepsConv[3],
                                           conv_extrap = FALSE),
                  naive1 = dWeibullCount_loglik(children, shape, scale,
                                                "conv_naive",
                                                1, TRUE, nstepsConv[4],
                                                conv_extrap = TRUE),
                  dePril0 = dWeibullCount_loglik(children, shape, scale,
                                                 "conv_dePril",
                                                 1, TRUE, nstepsConv[5],
                                                 conv_extrap = FALSE),
                  dePril1 = dWeibullCount_loglik(children, shape, scale,
                                                 "conv_dePril",
                                                 1, TRUE, nstepsConv[6],
                                                 conv_extrap = TRUE),
                  series_mat =
                      dWeibullCount_loglik(children, shape, scale,
                                           "series_mat", 1, TRUE,
                                           series_terms = ntermsSeries[1]),
                  series_acc =
                      dWeibullCount_loglik(children, shape, scale,
                                           "series_acc", 1, TRUE,
                                           series_terms = ntermsSeries[2],
                                           series_acc_eps = conv_series_acc),
                  mcShane = dWeibullCount_McShane(scale, shape,
                                                  children, jmax = 150),
                  replications = rep, order = "relative",
                  columns = c("test", "replications", "relative", "elapsed")
                  )

print(perf)
\end{verbatim}

\begin{center}
\begin{tabular}{lrrr}
series\(_{\text{acc}}\) & 1000 & 1 & 5.45999999999998\\
series\(_{\text{mat}}\) & 1000 & 1.31 & 7.15099999999995\\
direct1 & 1000 & 2.97 & 16.215\\
naive1 & 1000 & 2.98 & 16.273\\
dePril1 & 1000 & 3.158 & 17.241\\
dePril0 & 1000 & 11.139 & 60.82\\
naive0 & 1000 & 14.675 & 80.126\\
direct0 & 1000 & 16.381 & 89.441\\
mcShane & 1000 & 18.594 & 101.523\\
\end{tabular}
\end{center}

As can been seen, all the methods outperform the McShane's original code with
the series methods almost 20 times faster and the extrapolated convolution
methods roughly 6 times faster. As noted in the package documentation, the
series methods are less robust to large values of count and may fail for some
application. We therefore encourage the users to use the extrapolated
convolution method (the default in \texttt{Countr}) as much as possible.  

\label{sec:org0c77e0b}
We conclude this document by saving the work space to avoid re-running the
computation in future exportation of the document:
\begin{verbatim}
save.image()
\end{verbatim}




\label{sec:org3af36d9}
\bibliographystyle{apalike}
\bibliography{perf}
\end{document}